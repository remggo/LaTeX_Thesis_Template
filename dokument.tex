% arara: pdflatex: { draft: on }
% arara: biber
% arara: makeglossaries
% arara: pdflatex: { draft: on }
% arara: pdflatex: { synctex: on }
% arara: clean: {files : [dokument.log, dokument.aux, dokument.dvi, dokument.lof, dokument.lot, dokument.bit, dokument.idx, dokument.glo, dokument.bbl, dokument.bcf, dokument.ilg, dokument.toc, dokument.ind, dokument.out, dokument.blg, dokument.fdb_latexmk, dokument.fls, dokument.acn, dokument.acr, dokument.alg, dokument.glg, dokument.gls, dokument.glsdefs, dokument.ist, dokument.lol, dokument.run.xml, dokument.tdo]}
\documentclass[a4paper, 12pt, oneside, openright]{report}

\usepackage[ngerman,utf8]{sty/seminararbeit}

\makeglossaries
%--------------------------------------------------------------------------
% AKRONYME
%--------------------------------------------------------------------------
\newacronym{zaf}{ZAF}{Zentrum für angewandte Forschung}
\newacronym{cpu}{CPU}{Central Processing Unit}
\newacronym{thi}{THI}{Technischen Hochschule Ingolstadt}
\newacronym{oth}{OTH Regensburg}{Ostbayrische Technische Hochschule Regensburg}
\newacronym{cet}{CET}{Core Execution Time}
\newacronym{wcet}{WCET}{Worst Case Execution Time}
\newacronym{plpp}{PLPP}{Pattern Language for Parallel Programming}
\newacronym{numa}{NUMA}{Non-Uniform Memory Access}
\newacronym{embb}{EMBB}{Embedded Multicore Building Blocks}
\newacronym{mtapi}{MTAPI}{Multicore Task Management \gls{api}}
\newacronym{api}{API}{Application Programming Interface} 
\newacronym{ast}{AST}{Abstract Syntax Tree} 
\newacronym{arcs}{ARCS}{Architecture of Computing Systems} 
%--------------------------------------------------------------------------
% GLOSSAREINTRÄGE
%--------------------------------------------------------------------------

% Counter um die Römischen Seiten durchlaufen zu lassen
\newcounter{seitenZahl}

\addbibresource{references.bib} % Referenzieren der der Quellen

\begin{document}
    \listoftodos
	
    \setcounter{page}{0}
	\pagenumbering{roman}
	\gentitlepage{pics/Innen-square}{logos/thi_FEI_logo_wb_CMYK}{\scalebox{1.2}{Remko van Wagensveld}}
			{\scalebox{1}{Titel}\\\scalebox{1}{Untertitel}}


	%-----------------------------------------------------------------------------------------------
	% Schmutztitelblatt
	%-----------------------------------------------------------------------------------------------		
	\thispagestyle{empty}
\begin{bottompar}
	\begin{tabular}{ll}
		\textbf{Autor} & Remko van Wagensveld \\
		\textbf{Matrikelnummer} & 54376 \\
		\textbf{Abgabedatum} & 01.12.2015\\
		\textbf{Fach} & Professionelle Textsatzsysteme \\ 
		\textbf{Studiengang} & Informatik Bachelor \\
		\textbf{Fakultät} & Elektrotechnik und Informatik \\
		\textbf{Prüfer} & Dr. Paul Spannaus\\
		\textbf{Coverbild} & Ad van Wagensveld \cite{VW.2015}
	\end{tabular}	
\end{bottompar}
	
	

	
	% Erklärung
\chapter*{Erklärung}
Hiermit erkläre ich, dass ich die vorliegende Seminararbeit bis auf die offizielle Betreuung durch den Aufgabensteller selbstständig und ohne fremde Hilfe verfasst habe.\absatz
Die verwendeten Quellen sowie die verwendeten Hilfsmittel sind vollständig angegeben. Wörtlich übernommene Textteile und übernommene Bilder und Zeichnungen sind in jedem Einzelfall kenntlich gemacht. \\[10ex]
Ingolstadt, 1. Dezember 2015
\addcontentsline{toc}{chapter}{Erklärung}
	
	\tableofcontents
	
	\setcounter{seitenZahl}{\arabic{page}}
	
	% TEXT ----------------------------------

	\printbibliography
	\addcontentsline{toc}{chapter}{Literaturverzeichnis}
	
	
	\pagenumbering{roman}
	% Seitenzahlen gehen da weiter wo das Inhaltsverzeichnis aufgehört hat
	\stepcounter{seitenZahl}
	\setcounter{page}{\theseitenZahl}

	\listoffigures 
	\begingroup
		% Keine neue Seite Anfangen
		\let\clearpage\relax
		\lstlistoflistings % Listingsverzeichnis
		\addcontentsline{toc}{chapter}{Listings} % Einfügen in das Inhaltsverzeichnis (geht nicht automatisch.)
		
		% Symbolverzeichnis umbennenn in Abkürzungsverzeichnis, da geeigneter in dieser Seminararbeit
		\printglossary[title=Glossar,toctitle=Glossar]
		\thispagestyle{empty} %Workaround um keine Kapitelzeile oben im Abkürzungsverzeichnis zu haben
	\endgroup
	
	% APPENDIX ----------------------------------
	
\end{document}
